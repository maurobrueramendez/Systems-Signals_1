
\documentclass[11pt,a4paper]{article}

% ---------- Packages ----------
\usepackage[margin=2cm]{geometry}
\usepackage{amsmath, amssymb, mathtools}
\usepackage{enumitem}
\usepackage{hyperref}
\usepackage{microtype}
\usepackage{tcolorbox}
\usepackage{fancyhdr}
\usepackage{titlesec}

% ---------- Page Style ----------
\pagestyle{fancy}
\fancyhf{}
\lhead{Signals \& Systems I}
\rhead{IIR Filters}
\cfoot{\thepage}

% ---------- Section Style ----------
\titleformat{\section}{\large\bfseries}{\thesection.}{0.5em}{}
\titleformat{\subsection}{\normalsize\bfseries}{\thesubsection}{0.5em}{}

\setlist[itemize]{noitemsep, topsep=2pt}
\setlist[enumerate]{noitemsep, topsep=2pt}

% ---------- Box Styles ----------
\newtcolorbox{theorybox}{
  colback=blue!3,
  colframe=blue!40,
  title=Theory,
  fonttitle=\bfseries,
  boxrule=0.6pt,
  arc=2pt
}

\newtcolorbox{methodbox}{
  colback=green!3,
  colframe=green!40,
  title=Method,
  fonttitle=\bfseries,
  boxrule=0.6pt,
  arc=2pt
}

\newtcolorbox{examplebox}{
  colback=gray!4,
  colframe=black!35,
  title=Example / Result,
  fonttitle=\bfseries,
  boxrule=0.6pt,
  arc=2pt
}

\begin{document}

\begin{center}
{\LARGE \textbf{IIR Filters — Seminar 9 \& 10}}\\[4pt]
{\large Professor-Style Theory Summary \& Cheat Sheet}
\end{center}

\section{Discrete-Time LTI Systems}

\begin{theorybox}
A causal discrete-time LTI system is described by a linear constant-coefficient difference equation (LCCDE):
\[
y[n] + \sum_{k=1}^{N} a_k y[n-k]
= \sum_{m=0}^{M} b_m x[n-m].
\]
Initial rest is assumed unless stated otherwise.
\end{theorybox}

\section{Z-Transform and System Function}

\begin{theorybox}
The $z$-transform of a sequence $x[n]$ is
\[
X(z) = \sum_{n=-\infty}^{\infty} x[n] z^{-n}.
\]
\end{theorybox}

\begin{methodbox}
To compute the system function $H(z)$:
\begin{enumerate}
\item Take the $z$-transform of the difference equation.
\item Use $z$-shift property: $\mathcal{Z}\{x[n-k]\} = z^{-k}X(z)$.
\item Solve for $H(z)=\dfrac{Y(z)}{X(z)}$.
\end{enumerate}
\end{methodbox}

\begin{examplebox}
\[
y[n] = a y[n-1] + x[n]
\quad\Rightarrow\quad
H(z)=\frac{1}{1-a z^{-1}}.
\]
\end{examplebox}

\section{Poles, Zeros, and Stability}

\begin{theorybox}
For
\[
H(z)=\frac{B(z)}{A(z)},
\]
zeros are roots of $B(z)$ and poles are roots of $A(z)$.
\end{theorybox}

\begin{theorybox}
A causal LTI system is BIBO stable if and only if all poles satisfy
\[
|p_i| < 1.
\]
Zeros do not affect stability.
\end{theorybox}

\section{Impulse Response and Convolution}

\begin{theorybox}
The impulse response $h[n]$ is the output of the system to $x[n]=\delta[n]$.
\end{theorybox}

\begin{methodbox}
Any input can be decomposed as:
\[
x[n] = \sum_{k=-\infty}^{\infty} x[k]\delta[n-k].
\]
By linearity and time invariance:
\[
y[n] = \sum_{k=-\infty}^{\infty} x[k]h[n-k].
\]
\end{methodbox}

\section{First-Order IIR Systems}

\begin{theorybox}
A first-order IIR system has the form:
\[
y[n]=a y[n-1]+b x[n].
\]
\end{theorybox}

\begin{examplebox}
Impulse response:
\[
h[n]=b a^n u[n], \qquad |a|<1.
\]
Step response:
\[
y[n]=\frac{1-a^{n+1}}{1-a}.
\]
\end{examplebox}

\section{Second-Order Resonators}

\begin{theorybox}
Complex conjugate poles at
\[
z = r e^{\pm j\omega_0}
\]
produce damped oscillations.
\end{theorybox}

\begin{methodbox}
The corresponding denominator polynomial is:
\[
A(z)=1-2r\cos(\omega_0)z^{-1}+r^2z^{-2}.
\]
\end{methodbox}

\begin{examplebox}
Impulse response:
\[
h[n]=C r^n\cos(\omega_0 n+\phi)u[n].
\]
\end{examplebox}

\section{Delayed Feedback Systems}

\begin{theorybox}
A delayed feedback system:
\[
y[n]=-a y[n-K]+x[n]
\]
has system function
\[
H(z)=\frac{1}{1+a z^{-K}}.
\]
\end{theorybox}

\begin{examplebox}
Poles satisfy:
\[
z^K=-a
\]
and are equally spaced in angle on a circle of radius $|a|^{1/K}$.
\end{examplebox}

\section{Exam Strategy}

\begin{methodbox}
\begin{enumerate}
\item Write the difference equation.
\item Compute $H(z)$.
\item Find poles and zeros.
\item Check stability.
\item Interpret time-domain behavior from pole locations.
\end{enumerate}
\end{methodbox}

\end{document}
