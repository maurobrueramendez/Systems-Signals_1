\documentclass[11pt,a4paper]{article}

% ---------- Packages ----------
\usepackage[margin=2.5cm]{geometry}
\usepackage{amsmath, amssymb, mathtools}
\usepackage{enumitem}
\usepackage{hyperref}
\usepackage{microtype}
\usepackage{tcolorbox}
\usepackage{fancyhdr}
\usepackage{titlesec}
\usepackage{marginnote}

% ---------- Page Style ----------
\pagestyle{fancy}
\fancyhf{}
\lhead{Signals \& Systems I}
\rhead{FIR Filters}
\cfoot{\thepage}

% ---------- Section Style ----------
\titleformat{\section}{\large\bfseries}{\thesection.}{0.5em}{}
\titleformat{\subsection}{\normalsize\bfseries}{\thesubsection}{0.5em}{}
\setlist[itemize]{noitemsep, topsep=2pt}
\setlist[enumerate]{noitemsep, topsep=2pt}

% ---------- Box Styles ----------
\newtcolorbox{theorybox}{
    colback=blue!3,
    colframe=blue!40,
    title=Theory,
    fonttitle=\bfseries,
    boxrule=0.6pt,
    arc=2pt
}

\newtcolorbox{methodbox}{
    colback=green!3,
    colframe=green!40,
    title=Method,
    fonttitle=\bfseries,
    boxrule=0.6pt,
    arc=2pt
}

\newtcolorbox{examplebox}{
    colback=gray!4,
    colframe=black!35,
    title=Example / Result,
    fonttitle=\bfseries,
    boxrule=0.6pt,
    arc=2pt
}

\begin{document}

\begin{center}
{\LARGE \textbf{FIR Filters — Seminar 6}}\\[4pt]
{\large Professor-Style Theory Summary \& Cheat Sheet}
\end{center}

% ============================================================
\section{FIR Systems and Impulse Responses}

\begin{theorybox}
A finite-impulse-response (FIR) discrete-time LTI system is defined by
\[
y[n] = \sum_{k=0}^{M} b_k\,x[n-k].
\]
The impulse response has \emph{finite} duration:
\[
h[n] = 0 \quad \text{for } n \notin \{0,\dots,M\}.
\]
Initial rest is assumed.
\marginnote{\footnotesize FIR = no feedback; response ends after $M$ samples.}
\end{theorybox}

\begin{examplebox}
For
\[
h[n] = \delta[n-1] - 2\delta[n-4],
\]
the difference equation is
\[
y[n] = x[n-1] - 2x[n-4].
\]
\end{examplebox}

% ============================================================
\section{Impulse Response Construction}

\begin{theorybox}
The impulse response is the output when
\[
x[n] = \delta[n].
\]
A shifted delta produces delayed samples:
\[
\delta[n-k] \Rightarrow h[k].
\]
\marginnote{\footnotesize Each delta term reveals one tap.}
\end{theorybox}

\begin{methodbox}
To obtain $h[n]$ from a difference equation:
\begin{enumerate}
\item Set $x[n]=\delta[n]$.
\item Substitute into $y[n] = \sum b_k x[n-k]$.
\item Read off nonzero samples at $n=k$.
\end{enumerate}
\end{methodbox}

\begin{examplebox}
For
\[
y[n]=x[n] + 2x[n-2] - 4.5x[n-3] + 5x[n-5],
\]
the impulse response is
\[
h[n]=\delta[n]+2\delta[n-2]-4.5\delta[n-3]+5\delta[n-5].
\]
\end{examplebox}

% ============================================================
\section{Convolution Representation}

\begin{theorybox}
Any input can be decomposed as
\[
x[n] = \sum_{k=-\infty}^{\infty} x[k]\delta[n-k].
\]
By linearity \& time invariance:
\[
y[n] = \sum_{k=-\infty}^{\infty} x[k]\,h[n-k].
\]
\marginnote{\footnotesize Output = weighted, shifted copies of $h[n]$.}
\end{theorybox}

\begin{methodbox}
Convolution with FIR:
\begin{enumerate}
\item Identify nonzero $h[k]$.
\item For each $k$, form $x[n-k]$.
\item Multiply by $h[k]$ and sum.
\end{enumerate}
\end{methodbox}

\begin{examplebox}
Two-tap FIR:
\[
h[n]=\delta[n]-2\delta[n-3].
\]
Output:
\[
y[n]=x[n]-2x[n-3].
\]
\end{examplebox}

% ============================================================
\section{Block Diagrams of FIR Filters}

\begin{theorybox}
An $M$-tap FIR filter uses:
\[
M \text{ delays}, \quad M+1 \text{ multipliers}, \quad \text{adders}.
\]
A $k$-sample delay implements $x[n-k]$.
\marginnote{\footnotesize Each tap corresponds to one coefficient $b_k$.}
\end{theorybox}

\begin{methodbox}
To draw the structure:
\begin{enumerate}
\item Create delay chain: $x[n] \rightarrow z^{-1} \rightarrow z^{-1} \dots$.
\item Tap outputs at $x[n-k]$.
\item Multiply each tap by $b_k$.
\item Sum all branches into $y[n]$.
\end{enumerate}
\end{methodbox}

\begin{examplebox}
For coefficients $(1,0,2,-4.5,0,5)$:
Delays produce $x[n], x[n-1],\dots,x[n-5]$; taps at $k=0,2,3,5$.
\end{examplebox}

% ============================================================
\section{Frequency Response of FIR Filters}

\begin{theorybox}
For an FIR filter,
\[
H(e^{j\omega}) = \sum_{k=0}^{M} b_k e^{-j\omega k}.
\]
Complex exponentials are eigenfunctions:
\[
x[n] = C e^{j\omega n} \Rightarrow y[n] = H(e^{j\omega})\,C e^{j\omega n}.
\]
\marginnote{\footnotesize Filter acts as gain \& phase shift at frequency $\omega$.}
\end{theorybox}

\begin{methodbox}
To compute $y[n]$ for $x[n]=C e^{j\omega n}$:
\begin{enumerate}
\item Evaluate $H(e^{j\omega}) = \sum b_k e^{-j\omega k}$.
\item Multiply amplitude: $A = |H|\,|C|$.
\item Add phase: $\varphi = \arg(C)+\arg(H)$.
\item Output: $y[n] = A e^{j(\omega n + \varphi)}$.
\end{enumerate}
\end{methodbox}

\begin{examplebox}
Given
\[
y[n]=2x[n]+4x[n-1]+2x[n-2],
\quad x[n]=11e^{j(0.3\pi n+0.5\pi)},
\]
\[
H(e^{j0.3\pi}) = 2 + 4e^{-j0.3\pi} + 2e^{-j0.6\pi},
\]
\[
y[n] = H(e^{j0.3\pi})\, 11 e^{j(0.3\pi n + 0.5\pi)}.
\]
\end{examplebox}

% ============================================================
\section{Sinusoids Through FIR Filters}

\begin{theorybox}
A cosine can be written as:
\[
A\cos(\omega n + \phi)
= \Re\{A e^{j(\omega n + \phi)}\}.
\]
FIR filters preserve frequency; amplitude and phase are modified by $H(e^{j\omega})$.
\end{theorybox}

\begin{methodbox}
To obtain output for a cosine input:
\begin{enumerate}
\item Convert to complex exponential.
\item Multiply by $H(e^{j\omega})$.
\item Convert back to cosine using real part.
\end{enumerate}
\end{methodbox}

\begin{examplebox}
Input:
\[
x[n]=A\cos(\omega n+\phi).
\]
Output:
\[
y[n]=|H(e^{j\omega})|A
\cos\!\big(\omega n+\phi+\arg H(e^{j\omega})\big).
\]
\end{examplebox}

% ============================================================
\section{Time Shifts and Scaling}

\begin{theorybox}
Time shift:
\[
x[n-n_0] \Rightarrow y[n-n_0].
\]
Scaling:
\[
\alpha x[n] \Rightarrow \alpha y[n].
\]
\marginnote{\footnotesize LTI property used repeatedly in FIR exercises.}
\end{theorybox}

\begin{examplebox}
If $y_1[n]$ is output for $x_1[n]$, then
\[
x_2[n] = \alpha x_1[n-n_0]
\Rightarrow
y_2[n] = \alpha y_1[n-n_0].
\]
\end{examplebox}

% ============================================================
\end{document}
