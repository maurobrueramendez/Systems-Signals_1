\documentclass[11pt,a4paper]{article}

% ---------- Packages ----------
\usepackage[margin=2.5cm]{geometry}
\usepackage{amsmath, amssymb, mathtools}
\usepackage{enumitem}
\usepackage{hyperref}
\usepackage{microtype}
\usepackage{tcolorbox}
\usepackage{fancyhdr}
\usepackage{titlesec}

% ---------- Page Style ----------
\pagestyle{fancy}
\fancyhf{}
\lhead{Signals \& Systems I}
\rhead{Z-Transform (Seminar 8)}
\cfoot{\thepage}

% ---------- Section Style ----------
\titleformat{\section}{\large\bfseries}{\thesection.}{0.5em}{}
\titleformat{\subsection}{\normalsize\bfseries}{\thesubsection}{0.5em}{}

\setlist[itemize]{noitemsep, topsep=2pt}
\setlist[enumerate]{noitemsep, topsep=2pt}

% ---------- Box Styles ----------
\newtcolorbox{theorybox}{
  colback=blue!3,
  colframe=blue!40,
  title=Theory,
  fonttitle=\bfseries,
  boxrule=0.6pt,
  arc=2pt
}

\newtcolorbox{methodbox}{
  colback=green!3,
  colframe=green!40,
  title=Method,
  fonttitle=\bfseries,
  boxrule=0.6pt,
  arc=2pt
}

\newtcolorbox{examplebox}{
  colback=gray!4,
  colframe=black!35,
  title=Example / Result,
  fonttitle=\bfseries,
  boxrule=0.6pt,
  arc=2pt
}

\begin{document}

\begin{center}
{\LARGE \textbf{Z-Transform — Seminar 8}}\\[4pt]
{\large Professor-Style Theory Summary \& Cheat Sheet}
\end{center}

% ==========================================================
\section{Z-Transform \& ROC}
% ==========================================================

\begin{theorybox}
Bilateral $z$-transform:
\[
X(z)=\sum_{n=-\infty}^{\infty} x[n]z^{-n}.
\]
ROC is a ring in the $z$-plane; it cannot include poles.\\
Unit circle must lie in ROC to define $H(e^{j\omega})$.
\end{theorybox}

\marginpar{\scriptsize ROC decides if frequency response exists}

\begin{examplebox}
Key pairs used in Seminar 8:
\[
\delta[n] \leftrightarrow 1,\qquad
a^n u[n] \leftrightarrow \frac{1}{1-az^{-1}},\ |z|>|a|.
\]
\end{examplebox}

% ==========================================================
\section{FIR System Functions}
% ==========================================================

\begin{theorybox}
FIR systems:
\[
H(z)=\sum_{k=0}^{M} h[k]z^{-k}
\quad\Longleftrightarrow\quad
y[n]=\sum_{k=0}^{M} h[k]x[n-k].
\]
All poles located at $z=0$ (multiplicity $M$). No stability concerns.
\end{theorybox}

\begin{methodbox}
To compute $H(z)$:
\begin{enumerate}
\item Write $h[n]$ explicitly (from difference form or exercise).
\item Apply $z$-transform using shifts $z^{-k}$.
\item Simplify into polynomial in $z^{-1}$.
\end{enumerate}
\end{methodbox}

\begin{examplebox}
Three-point moving average:
\[
h[n]=\tfrac13(\delta[n]+\delta[n-1]+\delta[n-2]),
\quad
H(z)=\tfrac13(1+z^{-1}+z^{-2}).
\]
\end{examplebox}

% ==========================================================
\section{Factorization \& Conjugate-Pair Expansion}
% ==========================================================

\begin{theorybox}
A factor $(1-a z^{-1})$ gives a zero at $z=a$.\\
Conjugate pair:
\[
(1-az^{-1})(1-a^*z^{-1})
=1-(a+a^*)z^{-1}+|a|^2 z^{-2}.
\]
\[
a=re^{j\theta}\Rightarrow a+a^*=2r\cos\theta.
\]
\end{theorybox}

\marginpar{\scriptsize Used in Exercises 2, 5}

\begin{methodbox}
Factor all terms as $(1-cz^{-1})$ or $(1\pm z^{-2})$ to identify:
\begin{itemize}
\item unit-circle zeros (nulls), 
\item radial zeros ($r<1$),
\item multiplicities.
\end{itemize}
\end{methodbox}

\begin{examplebox}
\[
(1-0.9e^{j\pi/3}z^{-1})(1-0.9e^{-j\pi/3}z^{-1})
=1-0.9z^{-1}+0.81z^{-2}.
\]
\end{examplebox}

% ==========================================================
\section{Frequency Response $H(e^{j\omega})$}
% ==========================================================

\begin{theorybox}
Evaluate on unit circle:
\[
H(e^{j\omega})=H(z)\big|_{z=e^{j\omega}}.
\]
If $H(e^{j\omega_0})=0$, the system cancels a sinusoid of frequency $\omega_0$.
\end{theorybox}

\begin{methodbox}
To compute $H(e^{j\omega})$:
\begin{enumerate}
\item Substitute $z=e^{j\omega}$, $z^{-1}=e^{-j\omega}$.
\item Factor out $e^{-j\alpha\omega}$ if needed.
\item Reduce expressions with $e^{j\theta}\pm e^{-j\theta}$.
\end{enumerate}
\end{methodbox}

\begin{examplebox}
For $H(z)=1-z^{-4}$:
\[
H(e^{j\omega})=1-e^{-j4\omega}
=2j\sin(2\omega)e^{-j2\omega}.
\]
Magnitude:
\[
|H(e^{j\omega})|=2|\sin(2\omega)|.
\]
Nulls at $\omega=\frac{k\pi}{2}$.
\end{examplebox}

% ==========================================================
\section{Sinusoidal Steady-State Response}
% ==========================================================

\begin{theorybox}
Input:
\[
x[n]=A\cos(\omega_0 n+\phi).
\]
Output of an LTI system:
\[
y[n]=A|H(e^{j\omega_0})|
\cos(\omega_0 n+\phi+\angle H(e^{j\omega_0})).
\]
\end{theorybox}

\marginpar{\scriptsize Amplitude + phase set by $H(e^{j\omega_0})$}

\begin{examplebox}
If $H(e^{j\pi/2})=0$, all components at $\omega=\pi/2$ vanish.
\end{examplebox}

% ==========================================================
\section{Impulse Response Patterns in Seminar 8}
% ==========================================================

\begin{theorybox}
Impulse response is obtained by inverse $z$-transform:
\[
H(z)=\sum h[k]z^{-k}
\quad\Longrightarrow\quad
h[n]=\sum h[k]\delta[n-k].
\]
FIR structures yield finite support.
\end{theorybox}

\begin{examplebox}
\[
H(z)=1 - z^{-4}
\quad\Rightarrow\quad
h[n]=\delta[n]-\delta[n-4].
\]
\end{examplebox}

\begin{examplebox}
\[
H(z)=\tfrac13(1+z^{-1}+z^{-2})
\Rightarrow
h[n]=\tfrac13(\delta[n]+\delta[n-1]+\delta[n-2]).
\]
\end{examplebox}

% ==========================================================
\section{Cascade Decomposition (H = H\_1H\_2)}
% ==========================================================

\begin{theorybox}
If $H(z)=H_1(z)H_2(z)$,
\[
x[n]\xrightarrow{H_1} w[n]\xrightarrow{H_2} y[n].
\]
Choice of $H_1$ can simplify intermediate signals.
\end{theorybox}

\marginpar{\scriptsize Used in Exercise 5}

\begin{methodbox}
\begin{enumerate}
\item Choose $H_1(z)$ producing a simple time-domain operation\\
(e.g.\ $1-z^{-4}$ gives $w[n]=x[n]-x[n-4]$).
\item Let $H_2(z)=H(z)/H_1(z)$.
\item Expand $H_2(z)$ into FIR form to write $y[n]$.
\end{enumerate}
\end{methodbox}

\begin{examplebox}
If
\[
H(z)=(1-z^{-4})(1-0.8e^{j\pi/4}z^{-1})(1-0.8e^{-j\pi/4}z^{-1}),
\]
choose
\[
H_1(z)=1-z^{-4},
\quad w[n]=x[n]-x[n-4].
\]
Then
\[
H_2(z)=1-0.8\sqrt{2}\,z^{-1}+0.64 z^{-2}.
\]
Thus
\[
y[n]=w[n]-0.8\sqrt{2}\,w[n-1]+0.64 w[n-2].
\]
\end{examplebox}

% ==========================================================
\section{Standard FIR Examples Used in Seminar 8}
% ==========================================================

\begin{examplebox}
Three-point average:
\[
H(z)=\tfrac13(1+z^{-1}+z^{-2}),
\quad
|H(e^{j\omega})|=\tfrac{|1+2\cos\omega|}{3}.
\]
\end{examplebox}

\begin{examplebox}
Two-point difference:
\[
H(z)=1-z^{-1},
\quad
|H(e^{j\omega})|=2|\sin(\omega/2)|.
\]
\end{examplebox}

\begin{examplebox}
Four-sample difference:
\[
H(z)=1-z^{-4},
\quad
|H(e^{j\omega})|=2|\sin(2\omega)|.
\]

\end{examplebox}

% ==========================================================
\section{Exam Strategy (Seminar 8)}
% ==========================================================

\begin{methodbox}
\begin{enumerate}
\item Factor $H(z)$ fully; identify all zeros and their angles.
\item Check if unit circle is in ROC before evaluating $H(e^{j\omega})$.
\item For sinusoidal inputs, evaluate $H(e^{j\omega_0})$.
\item Detect nulling frequencies via unit-circle zeros.
\item For cascades, pick $H_1(z)$ producing simple shifts.
\item Express final outputs in exact sinusoidal/frequency form.
\end{enumerate}
\end{methodbox}

\end{document}
