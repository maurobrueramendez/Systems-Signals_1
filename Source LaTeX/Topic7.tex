\documentclass[11pt,a4paper]{article}

% ---------- Packages ----------
\usepackage[margin=3cm]{geometry}
\usepackage{amsmath, amssymb, mathtools}
\usepackage{enumitem}
\usepackage{hyperref}
\usepackage{microtype}
\usepackage{tcolorbox}
\usepackage{fancyhdr}
\usepackage{titlesec}
\usepackage{marginnote}

% ---------- Page Style ----------
\pagestyle{fancy}
\fancyhf{}
\lhead{Signals \& Systems I}
\rhead{FIR Filters}
\cfoot{\thepage}

% ---------- Section Style ----------
\titleformat{\section}{\large\bfseries}{\thesection.}{0.5em}{}
\titleformat{\subsection}{\normalsize\bfseries}{\thesubsection}{0.5em}{}
\setlist[itemize]{noitemsep, topsep=2pt}
\setlist[enumerate]{noitemsep, topsep=2pt}

% ---------- Box Styles (Identical to Reference) ----------
\newtcolorbox{theorybox}{
    colback=blue!3,
    colframe=blue!40,
    title=Theory,
    fonttitle=\bfseries,
    boxrule=0.6pt,
    arc=2pt
}
\newtcolorbox{methodbox}{
    colback=green!3,
    colframe=green!40,
    title=Method,
    fonttitle=\bfseries,
    boxrule=0.6pt,
    arc=2pt
}
\newtcolorbox{examplebox}{
    colback=gray!4,
    colframe=black!35,
    title=Example / Result,
    fonttitle=\bfseries,
    boxrule=0.6pt,
    arc=2pt
}

% ---------- Document ----------
\begin{document}

\begin{center}
{\LARGE \textbf{Seminar 7 — FIR Filters \& Frequency Response}}\\[4pt]
{\large Professor-Style Theory Summary \& Cheat Sheet}
\end{center}

% ============================================================
\section{FIR Systems and Convolution}
% ============================================================

\begin{theorybox}
A finite impulse response (FIR) LTI system of order $M$ is defined by:
\[
y[n] = \sum_{k=0}^{M} b_k x[n-k],
\qquad
h[n] = b_n.
\]
Impulse response is finite-length, causal if $h[n]=0$ for $n<0$.
\end{theorybox}

\marginnote{\footnotesize FIR = no feedback, always BIBO stable.}

\begin{methodbox}
Computing the output:
\[
y[n] = \sum_{k=0}^{M} b_k x[n-k].
\]
\begin{enumerate}
\item Identify $b_k$ from the given expression or $H(e^{j\omega})$.
\item Use direct convolution for time-domain signals.
\end{enumerate}
\end{methodbox}

\begin{examplebox}
If $H(e^{j\omega}) = 1 - e^{-j\omega}$ then $h[n] = \delta[n] - \delta[n-1]$ and
\[
y[n] = x[n] - x[n-1].
\]
\end{examplebox}

% ============================================================
\section{Frequency Response of FIR Filters}
% ============================================================

\begin{theorybox}
The frequency response is
\[
H(e^{j\omega}) = \sum_{k=0}^{M} b_k e^{-j\omega k}.
\]
A complex exponential $e^{j\omega_0 n}$ is an eigenfunction:
\[
y[n] = H(e^{j\omega_0}) e^{j\omega_0 n}.
\]
\end{theorybox}

\marginnote{\footnotesize Magnitude = gain, phase = shift.}

\begin{methodbox}
To compute $H(e^{j\omega})$:
\begin{enumerate}
\item Write the FIR expansion $\sum b_k e^{-j\omega k}$.
\item Factor exponentials to express in magnitude--phase form.
\item For cascade systems: multiply responses $H=H_1H_2H_3$.
\end{enumerate}
\end{methodbox}

\begin{examplebox}
For $y[n]=x[n]-\sqrt{3}\,x[n-1]+x[n-2]$,
\[
H(e^{j\omega}) = 1 - \sqrt{3}\,e^{-j\omega} + e^{-j2\omega}
= e^{-j\omega}(2\cos\omega -\sqrt{3}).
\]
\end{examplebox}

% ============================================================
\section{Zeros of FIR Systems}
% ============================================================

\begin{theorybox}
Zeros of $H(e^{j\omega})$ occur when the polynomial
\[
\sum_{k=0}^{M} b_k e^{-j\omega k} = 0.
\]
A sinusoid at frequency $\omega_0$ is fully cancelled if
\[
H(e^{j\omega_0}) = 0.
\]
\end{theorybox}

\marginnote{\footnotesize Zeros create notches in frequency.}

\begin{methodbox}
Finding zeros:
\begin{enumerate}
\item Rewrite $H(e^{j\omega})$ as $e^{-jM\omega/2} P(\omega)$.
\item Solve $P(\omega)=0$ using trigonometric identities.
\item For cascaded systems: each factor yields its own zeros.
\end{enumerate}
\end{methodbox}

\begin{examplebox}
If $H(e^{j\omega}) = e^{-j\omega}(2\cos\omega -\sqrt{3})$ then
\[
H(e^{j\omega}) = 0 \;\Longleftrightarrow\; 2\cos\omega = \sqrt{3}
\;\Longleftrightarrow\; \omega = \pm \tfrac{\pi}{6}.
\]
\end{examplebox}

% ============================================================
\section{Cascade of FIR Filters}
% ============================================================

\begin{theorybox}
For FIR systems in cascade:
\[
H(e^{j\omega}) = \prod_{i} H_i(e^{j\omega}), \qquad
h[n] = h_1[n] * h_2[n] * \cdots.
\]
\end{theorybox}

\marginnote{\footnotesize Cascading adds delays, multiplies spectra.}

\begin{methodbox}
To find the equivalent difference equation:
\begin{enumerate}
\item Compute each $H_i(e^{j\omega})$.
\item Multiply: $H=\prod H_i$.
\item Expand into $\sum b_k e^{-j\omega k}$.
\item Read off $b_k$ for the final FIR.
\end{enumerate}
\end{methodbox}

\begin{examplebox}
Cascading
$(1-e^{-j\omega})(1-e^{-j2\omega})(1-e^{-j3\omega})$ yields
\[
H(e^{j\omega}) = 1 - e^{-j\omega} - e^{-j2\omega}
+ e^{-j4\omega} + e^{-j5\omega} - e^{-j6\omega}.
\]
\end{examplebox}

% ============================================================
\section{Sinusoidal Response}
% ============================================================

\begin{theorybox}
For input $x[n]=A e^{j(\omega_0 n+\phi)}$,
\[
y[n] = A |H(e^{j\omega_0})|
e^{j(\omega_0 n+\phi+\angle H(e^{j\omega_0}))}.
\]
Frequency does not change; only amplitude and phase do.
\end{theorybox}

\marginnote{\footnotesize FIR acts as frequency-dependent scaling.}

\begin{methodbox}
To express output as $B\cos(\omega_0 n+\varphi)$:
\begin{enumerate}
\item Compute $H(e^{j\omega_0})$.
\item Let $B = A |H(e^{j\omega_0})|$.
\item Set $\varphi = \phi + \angle H(e^{j\omega_0}) \pm \tfrac{\pi}{2}$ depending on sine/cosine form.
\end{enumerate}
\end{methodbox}

\begin{examplebox}
If $x[n]=\sin(\omega_0 n)$ and
$H(e^{j\omega_0}) = e^{-j\omega_0}(2\cos\omega_0 -\sqrt{3})$,
\[
y[n] = (2\cos\omega_0 -\sqrt{3})
\cos(\omega_0 n -\omega_0 - \tfrac{\pi}{2}).
\]
\end{examplebox}

% ============================================================
\section{Moving-Average (Rectangular) FIR Filters}
% ============================================================

\begin{theorybox}
An $L$-point moving average:
\[
h[n] = \frac{1}{L} u[n] - \frac{1}{L} u[n-L],
\qquad 
H(e^{j\omega}) = \frac{1}{L}\sum_{k=0}^{L-1} e^{-j\omega k}.
\]
\end{theorybox}

\marginnote{\footnotesize Prototype low-pass; linear phase.}

\begin{methodbox}
Standard simplification:
\[
H(e^{j\omega}) =
\frac{1}{L}\,
e^{-j\omega\frac{L-1}{2}}
\frac{\sin(L\omega/2)}{\sin(\omega/2)}.
\]
Steps:
\begin{enumerate}
\item Use geometric series: $\sum e^{-j\omega k}$.
\item Factor exponentials to isolate linear phase.
\item Identify zeros at $\omega = \tfrac{2\pi m}{L}$.
\end{enumerate}
\end{methodbox}

\begin{examplebox}
Magnitude:
\[
|H(e^{j\omega})|
=
\frac{1}{L}
\left|\frac{\sin(L\omega/2)}{\sin(\omega/2)}\right|.
\]
Phase:
\[
\angle H(e^{j\omega}) = -\frac{L-1}{2}\omega
\quad (\text{plus sign flips at zeros}).
\]
\end{examplebox}

% ============================================================
\section{Exam Strategy}
% ============================================================

\begin{methodbox}
\begin{enumerate}
\item Identify FIR order and coefficients.
\item Compute $H(e^{j\omega})$ in exponential form.
\item Factor into magnitude--phase when useful.
\item Locate zeros: solve $H(e^{j\omega_0})=0$.
\item Evaluate system response to sinusoids via $H(e^{j\omega_0})$.
\item For cascades: multiply responses before expanding.
\end{enumerate}
\end{methodbox}

\end{document}
