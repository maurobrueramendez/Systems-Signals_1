\documentclass[11pt,a4paper]{article}

% ---------- Packages ----------
\usepackage[margin=2.5cm]{geometry}
\usepackage{amsmath, amssymb, mathtools}
\usepackage{enumitem}
\usepackage{hyperref}
\usepackage{microtype}
\usepackage{tcolorbox}
\usepackage{fancyhdr}
\usepackage{titlesec}

% ---------- Page Style ----------
\pagestyle{fancy}
\fancyhf{}
\lhead{Signals \& Systems I}
\rhead{Sinusoids \& Exponentials}
\cfoot{\thepage}

% ---------- Section Style ----------
\titleformat{\section}{\large\bfseries}{\thesection.}{0.5em}{}
\titleformat{\subsection}{\normalsize\bfseries}{\thesubsection}{0.5em}{}
\setlist[itemize]{noitemsep, topsep=2pt}
\setlist[enumerate]{noitemsep, topsep=2pt}

% ---------- Box Styles ----------
\newtcolorbox{theorybox}{
  colback=blue!3,
  colframe=blue!40,
  title=Theory,
  fonttitle=\bfseries,
  boxrule=0.6pt,
  arc=2pt
}

\newtcolorbox{methodbox}{
  colback=green!3,
  colframe=green!40,
  title=Method,
  fonttitle=\bfseries,
  boxrule=0.6pt,
  arc=2pt
}

\newtcolorbox{examplebox}{
  colback=gray!4,
  colframe=black!35,
  title=Example / Result,
  fonttitle=\bfseries,
  boxrule=0.6pt,
  arc=2pt
}

% ---------------------------------------------------------------
\begin{document}

\begin{center}
  {\LARGE \textbf{Sinusoids \& Exponentials --- Seminar 1 \& 2}}\\[4pt]
  {\large Professor-Style Theory Summary \& Cheat Sheet}
\end{center}

% ============================================================
\section{Continuous-Time Sinusoids}

\begin{theorybox}
A real sinusoid is defined as
\[
x(t) = A\cos(\omega_0 t + \varphi),
\]
with:
\[
A>0 \ (\text{amplitude}), \quad 
\omega_0>0 \ (\text{angular frequency [rad/s]}), \quad
t \in \mathbb{R}, \quad 
\varphi \in [0,2\pi).
\]
Fundamental period:
\[
T_0 = \frac{2\pi}{\omega_0}, \qquad
x(t+T_0)=x(t).
\]
Range:
\[
-A \le x(t) \le A.
\]
\end{theorybox}

\marginpar{\footnotesize Intuition: amplitude sets vertical scale; $\omega_0$ sets speed; $\varphi$ shifts horizontally.}

\begin{methodbox}
To determine sinusoid parameters from a plot:
\begin{enumerate}
  \item Amplitude $A$: half the peak-to-peak value.
  \item Period $T_0$: distance between successive peaks.
  \item Angular frequency: $\omega_0 = 2\pi/T_0$.
  \item Phase $\varphi$: solve $x(0)=A\cos(\varphi)$ or use time shift of max.
  \item Time shift relation:
  \[
  \cos(\omega_0 t + \varphi)=\cos(\omega_0(t-t_1)) \Rightarrow \varphi = -\omega_0 t_1.
  \]
\end{enumerate}
\end{methodbox}

\begin{examplebox}
From the sinusoid in Seminar 1 (Solutions p.2):
\[
A=9, \quad \omega_0 = 2\pi\frac{20}{9}, \quad \varphi = -2\pi\frac{1}{9}.
\]
\end{examplebox}

% ============================================================
\section{Sketching Sinusoids and Phase Shifts}

\begin{theorybox}
A horizontal shift of $x(t)=A\cos(\omega_0 t + \varphi)$ corresponds to
\[
x(t)=A\cos(\omega_0 (t-t_1)), \qquad \varphi = -\omega_0 t_1.
\]
Positive $t_1$ shifts the waveform right; negative shifts it left.
\end{theorybox}

\marginpar{\footnotesize Intuition: cosine achieves its max when argument $=0$ modulo $2\pi$.}

\begin{methodbox}
To plot shifted cosines:
\begin{enumerate}
  \item Identify base period $T_0 = 1/f_0$ or $2\pi/\omega_0$.
  \item Compute shift $t_1 = -\varphi/\omega_0$.
  \item Translate the standard cosine plot by $t_1$.
  \item Mark maxima at $t=t_1+kT_0$.
\end{enumerate}
\end{methodbox}

\begin{examplebox}
Seminar 1: For $\omega_0=\pi/5$ and $\varphi=-\pi/3$,
\[
t_1 = -\frac{-\pi/3}{\pi/5} = \frac{5}{3}.
\]
\end{examplebox}

% ============================================================
\section{Euler Representation and Phasors}

\begin{theorybox}
Euler identities:
\[
e^{j\theta} = \cos\theta + j\sin\theta,\qquad
\cos\theta = \frac{e^{j\theta}+e^{-j\theta}}{2},\qquad
\sin\theta = \frac{e^{j\theta}-e^{-j\theta}}{2j}.
\]
A phasor is the complex constant
\[
X = A e^{j\varphi}, \qquad x(t)=\Re\{X e^{j\omega_0 t}\}.
\]
\end{theorybox}

\marginpar{\footnotesize Intuition: sinusoid = rotating phasor projected onto real axis.}

\begin{methodbox}
Sum of sinusoids of same frequency:
\begin{enumerate}
  \item Convert each into phasor $X_k=A_k e^{j\varphi_k}$.
  \item Add phasors algebraically: $X=\sum_k X_k$.
  \item Resulting sinusoid:
  \[
  x(t)=|X|\cos(\omega_0 t + \arg X).
  \]
\end{enumerate}
\end{methodbox}

\begin{examplebox}
Seminar 2, Ex. 9:
\[
2\cos(200\pi t+\tfrac{\pi}{3}) + 2\cos(200\pi t-\tfrac{3\pi}{4})
\]
Phasors:
\[
X_1 = 2 e^{j\pi/3},\quad X_2 = 2 e^{-j3\pi/4},\quad 
X=X_1+X_2=5.536\, e^{j0.2747}.
\]
Thus,
\[
x(t)=5.536\cos(200\pi t+0.2747).
\]
\end{examplebox}

% ============================================================
\section{Trigonometric Identities from Euler}

\begin{theorybox}
Using Euler,
\[
\cos(\theta_1 \pm \theta_2)
= \cos\theta_1\cos\theta_2 \mp \sin\theta_1\sin\theta_2.
\]
\end{theorybox}

\marginpar{\footnotesize Intuition: product of phasors encodes angle sums.}

\begin{methodbox}
Procedure:
\begin{enumerate}
  \item Replace cosines/sines using Euler expressions.
  \item Multiply exponentials: $e^{j\theta_1}e^{j\theta_2}=e^{j(\theta_1+\theta_2)}$.
  \item Collect terms and recover cosine/sine via inverse Euler.
\end{enumerate}
\end{methodbox}

\begin{examplebox}
Seminar 2, Ex. 4:
\[
\cos(\theta_1+\theta_2)=\frac{e^{j\theta_1}+e^{-j\theta_1}}{2}
\cdot
\frac{e^{j\theta_2}+e^{-j\theta_2}}{2} \Rightarrow \cos\theta_1\cos\theta_2-\sin\theta_1\sin\theta_2.
\]
\end{examplebox}

% ============================================================
\section{Time Shift and Maximum Location}

\begin{theorybox}
For $x(t)=A\cos(\omega_0 t+\varphi)$, the maxima occur at
\[
\omega_0 t + \varphi = 2\pi k,\qquad t = \frac{-\varphi}{\omega_0} + kT_0.
\]
\end{theorybox}

\begin{methodbox}
To check maximum at time $t_1$:
\begin{enumerate}
  \item Set $t=t_1$ and check whether $\omega_0 t_1+\varphi \equiv 0 \pmod{2\pi}$.
  \item If yes, $x(t_1)=A$ is a maximum.
\end{enumerate}
\end{methodbox}

\begin{examplebox}
Seminar 2, Ex. 5:
\[
x(t)=A\sin(11\pi t)=A\cos(11\pi t - \tfrac{\pi}{2}).
\]
Thus $t_1 = \frac{1}{22}$ gives $\omega_0 t_1 - \pi/2 = 0$.
\end{examplebox}

% ============================================================
\section{Scaling and Shifting a Sinusoid}

\begin{theorybox}
If
\[
y(t)=Gx(t-t_1),
\]
and $x(t)=A\cos(\omega_0 t+\varphi)$, then
\[
y(t)=GA\cos(\omega_0 t+\varphi - \omega_0 t_1).
\]
\end{theorybox}

\marginpar{\footnotesize Intuition: time shift modifies phase; scaling modifies amplitude.}

\begin{methodbox}
To match $y(t)=B\cos(\omega_0 t)$:
\begin{enumerate}
  \item Set $GA=B$ for amplitude.
  \item Solve $\varphi-\omega_0 t_1=0$ for $t_1$.
\end{enumerate}
\end{methodbox}

\begin{examplebox}
Seminar 2, Ex. 6:
\[
x(t)=20\cos(80\pi t -0.4\pi).
\]
We want $y(t)=5\cos(80\pi t)$.
\[
G=\frac{5}{20}=\frac14,\qquad 
t_1=\frac{-(-0.4\pi)}{80\pi}=-\frac{1}{200}.
\]
\end{examplebox}

% ============================================================
\section{Phasor Sum Identity}

\begin{theorybox}
For $X_1=e^{j\alpha}$ and $X_2=e^{j\beta}$,
\[
X_3=X_1+X_2 = 2\cos\left(\frac{\alpha-\beta}{2}\right)\,
e^{j(\alpha+\beta)/2}.
\]
\end{theorybox}

\marginpar{\footnotesize Intuition: sum of two unit phasors is determined by midpoint angle and half-angle difference.}

\begin{examplebox}
Seminar 2, Ex. 7:
Plotting $X_1, X_2$, their vector sum has magnitude $2\cos((\alpha-\beta)/2)$ and angle $(\alpha+\beta)/2$.
\end{examplebox}

% ============================================================
\section{Discrete-Time Complex Exponentials}

\begin{theorybox}
A discrete complex exponential:
\[
x[n] = X z_0^n,
\]
with phasor $X=Ae^{j\varphi}$ and $z_0=re^{j\omega_0}$, is a discrete sinusoid when $r=1$.
\end{theorybox}

\marginpar{\footnotesize Intuition: repeated rotation/scaling in the complex plane.}

\begin{methodbox}
First difference:
\[
y[n]=x[n]-x[n-1]
\]
has form $Ae^{j(\omega_0 n + \phi)}$ when
\[
A = |1-z_0^{-1}||X|,\quad \phi = \arg(X(1-z_0^{-1})).
\]
\end{methodbox}

\begin{examplebox}
Seminar 2, Ex. 8:
\[
x[n]=e^{j(0.4\pi n - 0.5\pi)},\quad 
y[n]=x[n]-x[n-1]=A e^{j(\omega_0 n+\phi)}.
\]
Computed:
\[
A=1.37,\qquad \phi=-0.81,\qquad \omega_0=0.4\pi.
\]
\end{examplebox}

% ============================================================
\section{Complex-Valued $z(t)$ for Real $x(t)$}

\begin{theorybox}
Any real sinusoid can be expressed as
\[
x(t)=\Re\{z(t)\}, \qquad
z(t)=X e^{j\omega_0 t}.
\]
\end{theorybox}

\marginpar{\footnotesize Intuition: real part picks out projection of rotating phasor.}

\begin{examplebox}
Seminar 2, Ex. 10:
\[
x(t)=20\cos(300\pi t+\tfrac{\pi}{4})
+\!5\sqrt{2}\cos(300\pi t-\pi)
+\!5\sqrt{2}\cos(300\pi t-\tfrac{\pi}{2}).
\]
Phasor sum:
\[
X = 10 e^{j\pi/4},
\quad z(t)=X e^{j300\pi t},
\quad x(t)=10\cos(300\pi t+\tfrac{\pi}{4}).
\]
\end{examplebox}

% ============================================================
\section{Exam Strategy for Sinusoid Problems}

\begin{methodbox}
\begin{enumerate}
  \item Identify amplitude, period, frequency, and phase directly from plots or formulas.
  \item Convert between cosine and sine using phase shifts.
  \item For sums of sinusoids at same $\omega_0$:
  \begin{itemize}
    \item Rewrite each as a phasor.
    \item Add complex phasors.
    \item Convert back to single cosine.
  \end{itemize}
  \item For discrete exponentials:
  \begin{itemize}
    \item Identify $z_0$ and $X$.
    \item For differences, compute $1 - z_0^{-1}$.
    \item Extract amplitude and phase from magnitude and angle.
  \end{itemize}
  \item Always track angles in radians unless otherwise stated.
\end{enumerate}
\end{methodbox}

\end{document}
