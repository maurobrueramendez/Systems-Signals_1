\documentclass[11pt,a4paper]{article}

% ---------- Packages ----------
\usepackage[margin=2cm]{geometry}
\usepackage{amsmath, amssymb, mathtools}
\usepackage{enumitem}
\usepackage{hyperref}
\usepackage{microtype}
\usepackage{tcolorbox}
\usepackage{fancyhdr}
\usepackage{titlesec}

% ---------- Page Style ----------
\pagestyle{fancy}
\fancyhf{}
\lhead{Signals \& Systems I}
\rhead{Sampling — Seminar 5}
\cfoot{\thepage}

% ---------- Section Style ----------
\titleformat{\section}{\large\bfseries}{\thesection.}{0.5em}{}
\titleformat{\subsection}{\normalsize\bfseries}{\thesubsection}{0.5em}{}

\setlist[itemize]{noitemsep, topsep=2pt}
\setlist[enumerate]{noitemsep, topsep=2pt}

% ---------- Box Styles ----------
\newtcolorbox{theorybox}{
    colback=blue!3,
    colframe=blue!40,
    title=Theory,
    fonttitle=\bfseries,
    boxrule=0.6pt,
    arc=2pt
}
\newtcolorbox{methodbox}{
    colback=green!3,
    colframe=green!40,
    title=Method,
    fonttitle=\bfseries,
    boxrule=0.6pt,
    arc=2pt
}
\newtcolorbox{examplebox}{
    colback=gray!4,
    colframe=black!35,
    title=Example / Result,
    fonttitle=\bfseries,
    boxrule=0.6pt,
    arc=2pt
}

\begin{document}

\begin{center}
{\LARGE \textbf{Sampling — Seminar 5}}\\[4pt]
{\large Professor-Style Theory Summary \& Cheat Sheet}
\end{center}

% ============================================================
\section{Sampling Operator and Invertibility}

\begin{theorybox}
Sampling operator: $x[n] = x(nT_s)$ with $f_s = 1/T_s$.  
Invertibility requires a bandlimited signal with spectral support  
$|f| \le f_{\max}$.  
Nyquist–Shannon condition:  
\[
f_s > 2 f_{\max}.
\]
If $f_s < 2 f_{\max}$, different continuous signals may yield the same $x[n]$.
\end{theorybox}

\begin{methodbox}
To determine invertibility:
\begin{enumerate}
    \item Identify highest spectral component $f_{\max}$.
    \item Check Nyquist: $f_s > 2 f_{\max}$.
    \item If violated, enumerate aliases $f = \pm f_0 + m f_s$.
\end{enumerate}
\end{methodbox}

\begin{examplebox}
For $x(t)=4+4\cos(1000\pi t)\sin(50000\pi t)$, expansion gives  
frequencies $\{24.5,25.5\}\,\text{kHz}$ $\Rightarrow f_{\max}=25.5\,$kHz.  
Nyquist: $f_s > 51$ kHz.
\end{examplebox}

\emph{Intuition: two samples per period prevent ambiguity between oscillations.}

% ============================================================
\section{Spectral Expansion of Products}

\begin{theorybox}
Trig identity used repeatedly:
\[
\sin(a)\cos(b)=\frac{\sin(a+b)+\sin(a-b)}{2}.
\]
Frequencies extracted from $\cos(\omega t)$ and $\sin(\omega t)$ via  
$f=\omega/(2\pi)$.
\end{theorybox}

\begin{methodbox}
\begin{enumerate}
    \item Convert products into sum of sinusoids.
    \item Read frequencies directly from $\omega$.
    \item Determine $f_{\max}$ from resulting components.
\end{enumerate}
\end{methodbox}

\begin{examplebox}
$\cos(50\pi t)\sin(700\pi t)=\tfrac12[\sin(750\pi t)+\sin(650\pi t)]$  
$\Rightarrow f=\{375,325\}$ Hz.
\end{examplebox}

\emph{Intuition: product signals create sum/difference frequencies.}

% ============================================================
\section{Aliasing and Equivalence Classes}

\begin{theorybox}
Two continuous signals produce the same samples iff their frequencies satisfy
\[
f' = \pm f_0 + m f_s,\qquad m\in\mathbb{Z}.
\]
The discrete-time normalized frequency is  
\[
\Omega = 2\pi \frac{f}{f_s}.
\]
Aliasing occurs when different $f$ map to identical $\Omega$.
\end{theorybox}

\begin{methodbox}
\begin{enumerate}
    \item Compute base continuous frequency from $\Omega$: $f_0 = \Omega f_s / (2\pi)$.
    \item Enumerate aliases: $f = \pm f_0 + m f_s$.
    \item Retain those $|f|< f_s$ or within problem constraints.
\end{enumerate}
\end{methodbox}

\begin{examplebox}
For $x[n]=100\cos(0.4\pi n+\pi/4)$ with $f_s=5$ kHz:  
$f_0=(0.4\pi)/(2\pi)\cdot 5000=1000$ Hz.  
Aliases $<5$ kHz: $f_1=1000$ Hz, $f_2=4000$ Hz.
\end{examplebox}

\emph{Intuition: discrete-time frequencies ``wrap'' modulo $2\pi$.}

% ============================================================
\section{Continuous-to-Discrete Frequency Mapping}

\begin{theorybox}
Sampling transforms $\cos(2\pi f t+\phi)$ into  
\[
x[n]=\cos(2\pi f nT_s+\phi)=\cos(\Omega n+\phi),\qquad 
\Omega = 2\pi \frac{f}{f_s}.
\]
\end{theorybox}

\begin{methodbox}
\begin{enumerate}
    \item Replace $t$ by $nT_s$.
    \item Convert $2\pi f T_s$ to normalized $\Omega$.
    \item Reduce $\Omega$ modulo $2\pi$.
\end{enumerate}
\end{methodbox}

\begin{examplebox}
$x(t)=10+18\cos(140\pi t-\tfrac{2\pi}{3})$ sampled at $f_s=400$:  
\[
\Omega = 140\pi T_s = \frac{140\pi}{400}=0.35\pi.
\]
Thus $x[n]=10+18\cos(0.35\pi n - \tfrac{2\pi}{3})$.
\end{examplebox}

% ============================================================
\section{Minimum Sampling Rate}

\begin{theorybox}
Sampling without aliasing requires
\[
f_s > 2 f_{\max}.
\]
This ensures unique reconstruction and avoids overlapping shifted spectra.
\end{theorybox}

\begin{methodbox}
\begin{enumerate}
    \item Compute all sinusoidal frequencies of $x(t)$.
    \item Identify $f_{\max}$.
    \item Apply Nyquist inequality.
\end{enumerate}
\end{methodbox}

\begin{examplebox}
For $x(t)=\tfrac12[\sin(750\pi t)+\sin(650\pi t)]$:  
$f_{\max}=375$ Hz $\Rightarrow f_s>750$ Hz.
\end{examplebox}

\emph{Intuition: prevents spectral replicas from touching.}

% ============================================================
\section{Reconstruction from Spectral Lines}

\begin{theorybox}
Given spectral pairs  
$(A_k,f_k,\phi_k)$, the signal is  
\[
x(t)=\sum_k A_k \cos(2\pi f_k t+\phi_k).
\]
Hermitian symmetry ensures real-valuedness.
\end{theorybox}

\begin{methodbox}
\begin{enumerate}
    \item Read amplitude/phase from spectrum.  
    \item Convert angular frequencies $\omega_k$ to $f_k$.  
    \item Sum all cosine components.
\end{enumerate}
\end{methodbox}

\begin{examplebox}
From spectral points in Ex.~4:  
\[
x(t)=10+18\cos(140\pi t-\tfrac{2\pi}{3})+
10\cos(350\pi t+\tfrac{\pi}{2}).
\]
\end{examplebox}

% ============================================================
\section{Chirp Signals and Instantaneous Frequency}

\begin{theorybox}
For a phase $\theta(t)$, instantaneous frequency is
\[
f(t)=\frac{1}{2\pi}\frac{d\theta(t)}{dt}.
\]
Sampling constraints force $f(t)\le f_s/2$ (folding).
\end{theorybox}

\begin{methodbox}
\begin{enumerate}
    \item Replace $n$ by $t f_s$ when converting $x[n]$ to $x(t)$.  
    \item Differentiate $\theta(t)$ to obtain $f(t)$.  
    \item Clip at Nyquist if required.
\end{enumerate}
\end{methodbox}

\begin{examplebox}
For $\theta[n]=\pi10^{-4}n^2$ and $f_s=4$ kHz:  
$x(t)=\cos(1600\pi t^2)$,  
\[
f(t)=1600t.
\]
Folded spectrum limits $f(t)\le2000$ Hz.
\end{examplebox}

\emph{Intuition: chirps sweep frequency continuously; sampling restricts usable range.}

% ============================================================
\section{Phasor Interpretation}

\begin{theorybox}
Phasor: $z[n]=e^{j\theta[n]}$.  
If $\theta[n]=\Omega n+\phi$, rotation is uniform with period $N=2\pi/\Omega$.  
Nonlinear $\theta[n]$ yields non-periodic rotation (chirp phasor).
\end{theorybox}

\begin{methodbox}
\begin{enumerate}
    \item Compute angle $\theta[n]$.  
    \item Plot $e^{j\theta[n]}$ on unit circle.  
    \item Identify periodicity from $\theta[n+N]-\theta[n]=2\pi$.
\end{enumerate}
\end{methodbox}

\begin{examplebox}
$z[n]=e^{j(0.08\pi n-0.25\pi)}$ has period $25$.  
Chirp $c[n]=e^{j0.1\pi n^2}$ is non-periodic.
\end{examplebox}

\end{document}
